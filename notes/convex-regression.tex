\documentclass{article}

\usepackage{amsbsy,amsthm,amsmath,amssymb}
%\usepackage{graphicx,natbib,booktabs}
\usepackage{graphicx}
\usepackage[margin=1in]{geometry}
\usepackage{subfig}
\usepackage{algorithm}
\usepackage{algpseudocode}
\usepackage{paralist}

\newtheorem{proposition}{Proposition}
\newtheorem{lemma}{Lemma}
\newtheorem{example}{Example}
\newtheorem{definition}{Definition}
\def\E{\mathop{\rm E\,\!}\nolimits}
\def\Var{\mathop{\rm Var}\nolimits}
\def\Cov{\mathop{\rm Cov}\nolimits}
\def\den{\mathop{\rm den}\nolimits}
\def\midd{\mathop{\,|\,}\nolimits}
\def\sgn{\mathop{\rm sgn}\nolimits}
\def\vec{\mathop{\rm vec}\nolimits}
\def\sinc{\mathop{\rm sinc}\nolimits}
\def\curl{\mathop{\rm curl}\nolimits}
\def\div{\mathop{\rm div}\nolimits}
\def\tr{\mathop{\rm tr}\nolimits}
\def\len{\mathop{\rm len}\nolimits}
\def\dist{\mathop{\rm dist}\nolimits}
\def\prox{\mathop{\rm prox}\nolimits}
\def\amp{\mathop{\:\:\,}\nolimits}
\def\Real{\mathop{\mathbb{R}}\nolimits}
\def\dom{\mathop{\bf dom}\nolimits}
\def\argmin{\mathop{\rm argmin}\nolimits}
\def\argmax{\mathop{\rm argmax}\nolimits}
\newcommand{\svskip}{\vspace{1.75mm}}
\newcommand{\ba}{\boldsymbol{a}}
\newcommand{\bb}{\boldsymbol{b}}
\newcommand{\bc}{\boldsymbol{c}}
\newcommand{\bd}{\boldsymbol{d}}
\newcommand{\be}{\boldsymbol{e}}
\newcommand{\bff}{\boldsymbol{f}}
\newcommand{\bg}{\boldsymbol{g}}
\newcommand{\bh}{\boldsymbol{h}}
\newcommand{\bi}{\boldsymbol{i}}
\newcommand{\bj}{\boldsymbol{j}}
\newcommand{\bk}{\boldsymbol{k}}
\newcommand{\bl}{\boldsymbol{l}}
\newcommand{\bm}{\boldsymbol{m}}
\newcommand{\bn}{\boldsymbol{n}}
\newcommand{\bo}{\boldsymbol{o}}
\newcommand{\bp}{\boldsymbol{p}}
\newcommand{\bq}{\boldsymbol{q}}
\newcommand{\br}{\boldsymbol{r}}
\newcommand{\bs}{\boldsymbol{s}}
\newcommand{\bt}{\boldsymbol{t}}
\newcommand{\bu}{\boldsymbol{u}}
\newcommand{\bv}{\boldsymbol{v}}
\newcommand{\bw}{\boldsymbol{w}}
\newcommand{\bx}{\boldsymbol{x}}
\newcommand{\by}{\boldsymbol{y}}
\newcommand{\bz}{\boldsymbol{z}}
\newcommand{\bA}{\boldsymbol{A}}
\newcommand{\bB}{\boldsymbol{B}}
\newcommand{\bC}{\boldsymbol{C}}
\newcommand{\bD}{\boldsymbol{D}}
\newcommand{\bE}{\boldsymbol{E}}
\newcommand{\bF}{\boldsymbol{F}}
\newcommand{\bG}{\boldsymbol{G}}
\newcommand{\bH}{\boldsymbol{H}}
\newcommand{\bI}{\boldsymbol{I}}
\newcommand{\bJ}{\boldsymbol{J}}
\newcommand{\bK}{\boldsymbol{K}}
\newcommand{\bL}{\boldsymbol{L}}
\newcommand{\bM}{\boldsymbol{M}}
\newcommand{\bN}{\boldsymbol{N}}
\newcommand{\bO}{\boldsymbol{O}}
\newcommand{\bP}{\boldsymbol{P}}
\newcommand{\bQ}{\boldsymbol{Q}}
\newcommand{\bR}{\boldsymbol{R}}
\newcommand{\bS}{\boldsymbol{S}}
\newcommand{\bT}{\boldsymbol{T}}
\newcommand{\bU}{\boldsymbol{U}}
\newcommand{\bV}{\boldsymbol{V}}
\newcommand{\bW}{\boldsymbol{W}}
\newcommand{\bX}{\boldsymbol{X}}
\newcommand{\bY}{\boldsymbol{Y}}
\newcommand{\bZ}{\boldsymbol{Z}}
\newcommand{\balpha}{\boldsymbol{\alpha}}
\newcommand{\bbeta}{\boldsymbol{\beta}}
\newcommand{\bgamma}{\boldsymbol{\gamma}}
\newcommand{\bdelta}{\boldsymbol{\delta}}
\newcommand{\bepsilon}{\boldsymbol{\epsilon}}
\newcommand{\blambda}{\boldsymbol{\lambda}}
\newcommand{\bmu}{\boldsymbol{\mu}}
\newcommand{\bnu}{\boldsymbol{\nu}}
\newcommand{\bphi}{\boldsymbol{\phi}}
\newcommand{\bpi}{\boldsymbol{\pi}}
\newcommand{\bsigma}{\boldsymbol{\sigma}}
\newcommand{\btheta}{\boldsymbol{\theta}}
\newcommand{\bomega}{\boldsymbol{\omega}}
\newcommand{\bxi}{\boldsymbol{\xi}}
\newcommand{\bGamma}{\boldsymbol{\rho}}
\newcommand{\bDelta}{\boldsymbol{\Delta}}
\newcommand{\bTheta}{\boldsymbol{\Theta}}
\newcommand{\bLambda}{\boldsymbol{\Lambda}}
\newcommand{\bXi}{\boldsymbol{\Xi}}
\newcommand{\bPi}{\boldsymbol{\Pi}}
\newcommand{\bOmega}{\boldsymbol{\Omega}}
\newcommand{\bUpsilon}{\boldsymbol{\Upsilon}}
\newcommand{\bPhi}{\boldsymbol{\Phi}}
\newcommand{\bPsi}{\boldsymbol{\Psi}}
\newcommand{\bSigma}{\boldsymbol{\Sigma}}
%% Matrix-matrix operations
\newcommand{\Kron}{\otimes} %Kronecker
%\newcommand{\Khat}{\odot} %Khatri-Rao
\newcommand{\Hada}{\ast} %Hadamard
%\newcommand{\Divide}{\varoslash}
%% Norms
\newcommand{\abs}[1]{\lvert{#1}\rvert}


\title{First-Order Methods for Convex Multivariate Regression}
\author{Alfonso Landeros}
\date{\today}

\begin{document}
\maketitle

Convex regression is a nonparametric method to estimate a function under shape constraints.
Given a set of observations \(\{\bx_{i}, y_{i}\}_{i=1}^{n}\) with \(\bx_{i} \in \Real^{d}\) and \(y_{i} \in \Real\), our goal is determine the least squares estimator (LSE) \(\hat{\varphi}_{n}\) by solving
\begin{equation*}
    \hat{\varphi}_{n} \in \underset{\psi}{\arg \min}
    \sum_{i=1}^{n} (y_{i} - \psi(\bx_{i}))^{2}.
\end{equation*}
Here the minimum is taken over the set of all convex functions \(\psi : \Real^{d} \rightarrow \Real\).
The properties of the convex LSE has been described in detail \cite{seijo2011}.
The equivalent, finite dimensional problem
\begin{equation*}
    \underset{\btheta; \bxi_{1}, \ldots, \bxi_{n}}{\mathrm{minimize}}
    ~\frac{1}{2} \|\by - \btheta\|_{2}^{2}
    \quad
    \text{subject to}
    ~\theta_{j} + \langle{\bx_{i} - \bx_{j},~\bxi_{j}}\rangle \le \theta_{i}
\end{equation*}
estimates function values \(\theta_{i} = \varphi_{n}(\bx_{i})\) and subgradients \(\bxi_{i} \in \Real^{d}\) at each data point.
The discretization \(\hat{\btheta}\) can be extended to a continuous, convex function on \(\Real^{d}\) by the affine interpolation scheme
\begin{equation*}
    \hat{\varphi}(\bx)
    =
    \max_{j=1,\ldots,n}\{\hat{\theta}_{j}
    + \langle{\bx - \bx_{j},~\hat{\bxi}_{j}}\rangle\}.
\end{equation*}
Further, one can restore smoothness by imposing Lipschitz constraints on each subgradient or applying postprocessing techniques \cite{mazumder2018}.

Computation remains a challenge.
Generic interior point methods do not scale to large datasets due to the \(n^{2}\) constraints.
Alternating direction method of multipliers (ADMM) has been proposed as an alternative that adequately exploits problem-specific structure \cite{mazumder2018}.
Empirical evidence suggests ADMM scales to problems on the order of \(n \approx 5000\) data points and is largely insensitive to the dimension \(d\) of a function's domain.
However, the convergence properties of 3-block ADMM algorithms are not characterized and so the quality of solutions cannot be readily assessed.

These notes explore proximal gradient algorithms tailored to convex regression.
We anticipate that the simplicity of updates, acceleration techniques, and well-established convergence theory will prove superior to ADMM in this context.

\section*{\center A Proximal Algorithm}

The convex regression problem is a quadratic program
\begin{equation*}
    \underset{\bz}{\mathrm{minimize}}
    ~f(\bA \bz) \quad
    \text{subject to}
    ~\bB \bz \le \boldsymbol{0},
\end{equation*}
where \(\bz = (\btheta, \bxi_{1}, \ldots, \bxi_{n})^{t}\) and \(\bA\) and \(\bB\) are compatible matrices to be determined.
Its solution is approximated by an unconstrained program in which the sublevel set constraint is replaced with the algebraic penalty function \(\|(\bB \bz)_{+}\|_{2}^{2}\).
Here \(u_{+} = \max\{0, u\}\) is understood as an element-wise operation.
Explicitly, the unconstrained problem is of the form
\begin{equation*}
    \underset{\bz}{\mathrm{minimize}}
    ~\frac{1}{2} \|\by - \btheta\|^{2}_{2}
    + \frac{\rho}{2} \|(\bB \bz)_{+}\|_{2}^{2}.
\end{equation*}
Thus, the appropriate gradient is of the form
\begin{equation*}
    \bv_{n}
    =
    \bA^{\ast} \nabla f(\bA \bz) + \rho \bB^{\ast}(\bB \bz)_{+}
    =
    \bA^{\ast} \bA \bz + \rho \bB^{\ast} (\bB \bz)_{+},
\end{equation*}
and is used in the iterative map
\begin{equation*}
    \bz_{n+1} = \bz_{n} - \gamma_{n} \bv_{n}.
\end{equation*}
At each iteration, the optimal step size is given by
\begin{equation*}
    \gamma_{n}
    =
    \frac{\|\bv_{n}\|^{2}}{\|\bA \bv_{n}\|^{2} + \rho \|\bB \bv_{n}\|^{2}}.
\end{equation*}
These updates reduce to linear operations and straightforward projection onto a non-negative orthant.
Notably absent are costly matrix inversions.
Fast convergence and efficient computation hinge on the structure of \(\bA\) and \(\bB\).

\subsection*{Identifying missing matrices}

Our goal is to describe the original problem in terms of a single vector \(\bz = (\btheta, \bxi_{1}, \ldots, \bxi_{n})^{t}\).
Dealing with the loss term is straightforward:
\begin{equation*}
    \|\by - \btheta\|_{2}^{2}
    =
    \|\bA (\tilde{\by} - \bz)\|_{2}^{2},
\end{equation*}
with \(\bA = [\bI_{n \times n}~\boldsymbol{0}_{n \times d}~\ldots ~\boldsymbol{0}_{n \times d}]\) and \(\tilde{\by} = (\by, \boldsymbol{0}_{d \times 1}, \ldots, \boldsymbol{0}_{d \times 1})^{t}\).
Identifying \(\bB\) requires more work.
First, let \(\bDelta_{ij} = \bx_{i} - \bx_{j}\) and define \(C_{ij} = \langle{\bDelta_{ij}, \bxi_{j}}\rangle\).
It follows that column \(j\) of \(\bZ\) is given by
\begin{equation*}
    \bC_{\cdot, j}
    =
    \bDelta_{\cdot,j} \bxi_{j}
    =
    \begin{bmatrix}
        \bDelta_{1j}^{\ast} \\
        \bDelta_{2j}^{\ast} \\
        \vdots \\
        \bDelta_{nj}^{\ast}
    \end{bmatrix}
    \bxi_{j}
    =
    \begin{bmatrix}
        \langle\bDelta_{1j}, \bxi_{j}\rangle \\
        \langle\bDelta_{2j}, \bxi_{j}\rangle \\
        \vdots \\
        \langle\bDelta_{nj}, \bxi_{j}\rangle
    \end{bmatrix}_{n \times 1}.
\end{equation*}
The representation \(\bC = \sum_{j=1}^{n} \bC_{\cdot, j} \be_{j}^{\ast}\) then suggests that
\begin{equation*}
    \vec(\bC)
    =
    \sum_{j=1}^{n} (\be_{j} \otimes \bDelta_{\cdot,j})~\bxi_{j} 
    =
    \begin{bmatrix}
        (\be_{1} \otimes \bDelta_{\cdot,1})
        & (\be_{2} \otimes \bDelta_{\cdot,2})
        & \cdots
        & (\be_{n} \otimes \bDelta_{\cdot,n})
    \end{bmatrix}
    \begin{bmatrix}
        \bxi_{1} \\
        \bxi_{2} \\
        \vdots \\
        \bxi_{n}
    \end{bmatrix}.
\end{equation*}
Now observe that the \(n^{2}\) constraints can be described by the generalized inequality \(\bD \btheta + \vec(\bC) \le \boldsymbol{0}\), where we define \([\bD \btheta]_{(i-1)n+j} = \theta_{j} - \theta_{i}\) as an \(n^{2} \times n\) matrix.
Finally, combine our results to arrive at the identity
\begin{equation*}
    \bD \btheta + \vec(\bC)
    =
    \begin{bmatrix}
        \bD
        & (\be_{1} \otimes \bDelta_{\cdot,1})
        & (\be_{2} \otimes \bDelta_{\cdot,2})
        & \cdots
        & (\be_{n} \otimes \bDelta_{\cdot,n})
    \end{bmatrix}
    \begin{bmatrix}
        \btheta \\
        \bxi_{1} \\
        \bxi_{2} \\
        \vdots \\
        \bxi_{n}
    \end{bmatrix} \\
    =
    \bB \bz,
\end{equation*}
as promised at the beginning.

\subsection*{Exploiting structure to avoid explicit matrix operations}

Let \(\bz \in \Real^{n(1+d)}\) as before.
It is easy to check that \(\bA^{\ast} \bA \bz = (\btheta, \boldsymbol{0}, \ldots, \boldsymbol{0})\) as expected.
Fortunately, the matrix \(\bB \in \Real^{n^{2} \times nd}\) is also highly structured.
Computing \(\bB \bz\) is already straightforward based on the explicit form of the constraints.
However, the gradient contains the product \(\bB^{\ast} \bw\).
An explicit formula for this product would allow one to avoid forming any additional matrices altogether.
Such a feat saves on memory requirements and potentially accelerates computing \(\bB^{\ast} (\bB \bz)_{+}\).

Let
\(
\bH = \begin{bmatrix}
    \bH_{1} & \bH_{2} & \cdots & \bH_{n}
\end{bmatrix}
\)
with \(\bH_{j} = (\be_{j} \otimes \bDelta_{\cdot, j})\) so that
\(
\bB
=
\begin{bmatrix}
    \bD & \bH
\end{bmatrix}.
\)
The blocks of \(\bB^{\ast}\) will then correspond to updates to the variables \(\btheta, \bxi_{1}, \ldots, \bxi_{n}\).
We start by deriving an explicit formula for \(\bD \bw\)
\begin{equation*}
    \begin{split}
        \langle{\bD \btheta, \bw}\rangle
        &=
        \sum_{i,j} (\bD \btheta)_{(i-1)n+j} w_{(i-1)n+j} \\
        &=
        \sum_{i,j} (\btheta_{j} - \btheta_{i}) w_{ij} \\
        &=
        \langle{\bW^{\ast} \boldsymbol{1}, \btheta}\rangle
        -
        \langle{\bW \boldsymbol{1}, \btheta}\rangle \\
        &=
        \langle{(\bW^{\ast} - \bW) \boldsymbol{1}, \btheta}\rangle.
    \end{split}
\end{equation*}
Here it is understood that \(\bw = \vec(\bW)\).
We conclude \(\bD \bw = (\bW^{\ast} - \bW) \boldsymbol{1}\).
The products \(\bH_{j}^{\ast} \bw\) also have an explicit formula
\begin{equation*}
    \bH_{j}^{\ast} \bw
    =
    (\be_{j}^{\ast} \otimes \bDelta_{\cdot,j}^{\ast}) \bw
    =
    \bDelta_{\cdot,j}^{\ast} \bW \be_{j}
    =
    \sum_{i=1}^{n} \bDelta_{ij} w_{ij}
    =
    \sum_{i=1}^{n} w_{ij}(\bx_{i} - \bx_{j}).
\end{equation*}
The Gram matrix \(\bB^{\ast} \bB \in \Real^{n(1+d) \times n(1+d)}\) consists of the blocks
\begin{equation*}
    \begin{bmatrix}
        \bD^{\ast} \bD
        & \bD^{\ast} \bH_{1}
        & \bD^{\ast} \bH_{2}
        & \cdots
        & \bD^{\ast} \bH_{n} \\
        \bH_{1}^{\ast} \bD
        & \bDelta_{\cdot,1}^{\ast} \bDelta_{\cdot,1}
        & \boldsymbol{0}_{d \times d}
        & \cdots
        & \boldsymbol{0}_{d \times d} \\
        \bH_{2}^{\ast} \bD
        & \boldsymbol{0}_{d \times d}
        & \bDelta_{\cdot,2}^{\ast} \bDelta_{\cdot,2}
        & \ddots
        & \vdots \\
        \vdots
        & \vdots
        & \ddots
        & \ddots
        & \boldsymbol{0}_{d \times d} \\
       \bH_{n}^{\ast} \bD
        & \boldsymbol{0}_{d \times d}
        & \cdots
        & \boldsymbol{0}_{d \times d}
        &\bDelta_{\cdot,n}^{\ast} \bDelta_{\cdot,n}
    \end{bmatrix}.
\end{equation*}

Below we summarize the components of the gradient \(\bv_{n+1}\) at iteration \(n+1\):
\begin{align*}
    \bv_{\btheta}
    &=
    \by - \btheta_{n} + \rho [\bW^{\ast} - \bW] \boldsymbol{1} \\
    \bv_{\bxi_{j}}
    &=
    \rho \bDelta_{\cdot,j}^{\ast} \bW_{\cdot,j}.
\end{align*}
Here we have \(\bW = (\bB \bz_{n})_{+}\) encoding the value of each constraint after projecting to a non-negative orthant.
The proximal gradient updates take the form
\begin{align*}
    \btheta_{n+1}
    &=
    \btheta_{n}
    - \gamma_{n} [\by - \btheta_{n} + \rho (\bW^{\ast} - \bW) \boldsymbol{1}], \\
    \bxi_{n+1,j}
    &=
    \bxi_{n,j} - \gamma_{n} \rho \bDelta_{\cdot,j}^{\ast} \bW_{\cdot,j}.
\end{align*}

\section*{\center Simplifying with the MM Principle}

%%%%% BIBLIOGRAPHY %%%%%
\begin{thebibliography}{99}
    \bibitem{seijo2011}
    Seijo E, Sen B (2011) {Nonparametric least squares estimation of a multivariate convex regression function}. {\it The Annals of Statistics}. 1633-1657
    
    \bibitem{langeMM2016}
    Lange, K (2016) {MM Optimization Algorithms}. {\it SIAM}.

    \bibitem{mazumder2018}
    Mazumder R, Choudhury A, Iyengar G, Sen B (2018) {A computational framework for multivariate convex regression and its variants}. {\it Journal of the American Statistical Association}. 318-331
\end{thebibliography}
\end{document}