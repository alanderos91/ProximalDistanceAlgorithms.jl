\documentclass{article}

\usepackage{amsbsy,amsthm,amsmath,amssymb}
%\usepackage{graphicx,natbib,booktabs}
\usepackage{graphicx}
\usepackage[margin=1in]{geometry}
\usepackage{subfig}
\usepackage{algorithm}
\usepackage{algpseudocode}
\usepackage{paralist}

\newtheorem{proposition}{Proposition}
\newtheorem{lemma}{Lemma}
\newtheorem{example}{Example}
\newtheorem{definition}{Definition}
\def\E{\mathop{\rm E\,\!}\nolimits}
\def\Var{\mathop{\rm Var}\nolimits}
\def\Cov{\mathop{\rm Cov}\nolimits}
\def\den{\mathop{\rm den}\nolimits}
\def\midd{\mathop{\,|\,}\nolimits}
\def\sgn{\mathop{\rm sgn}\nolimits}
\def\vec{\mathop{\rm vec}\nolimits}
\def\sinc{\mathop{\rm sinc}\nolimits}
\def\curl{\mathop{\rm curl}\nolimits}
\def\div{\mathop{\rm div}\nolimits}
\def\tr{\mathop{\rm tr}\nolimits}
\def\len{\mathop{\rm len}\nolimits}
\def\dist{\mathop{\rm dist}\nolimits}
\def\prox{\mathop{\rm prox}\nolimits}
\def\amp{\mathop{\:\:\,}\nolimits}
\def\Real{\mathop{\mathbb{R}}\nolimits}
\def\dom{\mathop{\bf dom}\nolimits}
\def\argmin{\mathop{\rm argmin}\nolimits}
\def\argmax{\mathop{\rm argmax}\nolimits}
\newcommand{\svskip}{\vspace{1.75mm}}
\newcommand{\ba}{\boldsymbol{a}}
\newcommand{\bb}{\boldsymbol{b}}
\newcommand{\bc}{\boldsymbol{c}}
\newcommand{\bd}{\boldsymbol{d}}
\newcommand{\be}{\boldsymbol{e}}
\newcommand{\bff}{\boldsymbol{f}}
\newcommand{\bg}{\boldsymbol{g}}
\newcommand{\bh}{\boldsymbol{h}}
\newcommand{\bi}{\boldsymbol{i}}
\newcommand{\bj}{\boldsymbol{j}}
\newcommand{\bk}{\boldsymbol{k}}
\newcommand{\bl}{\boldsymbol{l}}
\newcommand{\bm}{\boldsymbol{m}}
\newcommand{\bn}{\boldsymbol{n}}
\newcommand{\bo}{\boldsymbol{o}}
\newcommand{\bp}{\boldsymbol{p}}
\newcommand{\bq}{\boldsymbol{q}}
\newcommand{\br}{\boldsymbol{r}}
\newcommand{\bs}{\boldsymbol{s}}
\newcommand{\bt}{\boldsymbol{t}}
\newcommand{\bu}{\boldsymbol{u}}
\newcommand{\bv}{\boldsymbol{v}}
\newcommand{\bw}{\boldsymbol{w}}
\newcommand{\bx}{\boldsymbol{x}}
\newcommand{\by}{\boldsymbol{y}}
\newcommand{\bz}{\boldsymbol{z}}
\newcommand{\bA}{\boldsymbol{A}}
\newcommand{\bB}{\boldsymbol{B}}
\newcommand{\bC}{\boldsymbol{C}}
\newcommand{\bD}{\boldsymbol{D}}
\newcommand{\bE}{\boldsymbol{E}}
\newcommand{\bF}{\boldsymbol{F}}
\newcommand{\bG}{\boldsymbol{G}}
\newcommand{\bH}{\boldsymbol{H}}
\newcommand{\bI}{\boldsymbol{I}}
\newcommand{\bJ}{\boldsymbol{J}}
\newcommand{\bK}{\boldsymbol{K}}
\newcommand{\bL}{\boldsymbol{L}}
\newcommand{\bM}{\boldsymbol{M}}
\newcommand{\bN}{\boldsymbol{N}}
\newcommand{\bO}{\boldsymbol{O}}
\newcommand{\bP}{\boldsymbol{P}}
\newcommand{\bQ}{\boldsymbol{Q}}
\newcommand{\bR}{\boldsymbol{R}}
\newcommand{\bS}{\boldsymbol{S}}
\newcommand{\bT}{\boldsymbol{T}}
\newcommand{\bU}{\boldsymbol{U}}
\newcommand{\bV}{\boldsymbol{V}}
\newcommand{\bW}{\boldsymbol{W}}
\newcommand{\bX}{\boldsymbol{X}}
\newcommand{\bY}{\boldsymbol{Y}}
\newcommand{\bZ}{\boldsymbol{Z}}
\newcommand{\balpha}{\boldsymbol{\alpha}}
\newcommand{\bbeta}{\boldsymbol{\beta}}
\newcommand{\bgamma}{\boldsymbol{\gamma}}
\newcommand{\bdelta}{\boldsymbol{\delta}}
\newcommand{\bepsilon}{\boldsymbol{\epsilon}}
\newcommand{\blambda}{\boldsymbol{\lambda}}
\newcommand{\bmu}{\boldsymbol{\mu}}
\newcommand{\bnu}{\boldsymbol{\nu}}
\newcommand{\bphi}{\boldsymbol{\phi}}
\newcommand{\bpi}{\boldsymbol{\pi}}
\newcommand{\bsigma}{\boldsymbol{\sigma}}
\newcommand{\btheta}{\boldsymbol{\theta}}
\newcommand{\bomega}{\boldsymbol{\omega}}
\newcommand{\bxi}{\boldsymbol{\xi}}
\newcommand{\bGamma}{\boldsymbol{\rho}}
\newcommand{\bDelta}{\boldsymbol{\Delta}}
\newcommand{\bTheta}{\boldsymbol{\Theta}}
\newcommand{\bLambda}{\boldsymbol{\Lambda}}
\newcommand{\bXi}{\boldsymbol{\Xi}}
\newcommand{\bPi}{\boldsymbol{\Pi}}
\newcommand{\bOmega}{\boldsymbol{\Omega}}
\newcommand{\bUpsilon}{\boldsymbol{\Upsilon}}
\newcommand{\bPhi}{\boldsymbol{\Phi}}
\newcommand{\bPsi}{\boldsymbol{\Psi}}
\newcommand{\bSigma}{\boldsymbol{\Sigma}}
%% Matrix-matrix operations
\newcommand{\Kron}{\otimes} %Kronecker
%\newcommand{\Khat}{\odot} %Khatri-Rao
\newcommand{\Hada}{\ast} %Hadamard
%\newcommand{\Divide}{\varoslash}
%% Norms
\newcommand{\abs}[1]{\lvert{#1}\rvert}


\title{Proximal Distance Application: Convex Multivariate Regression}
\author{Alfonso Landeros}
\date{\today}

\begin{document}
\maketitle

Convex regression is a nonparametric method to estimate a function under shape constraints.
Given a set of observations \(\{\bx_{i}, y_{i}\}_{i=1}^{n}\) with \(\bx_{i} \in \Real^{d}\) and \(y_{i} \in \Real\), our goal is determine the least squares estimator (LSE) \(\hat{\varphi}_{n}\) by solving
\begin{equation*}
    \hat{\varphi}_{n} \in \underset{\psi}{\arg \min}
    \sum_{i=1}^{n} (y_{i} - \psi(\bx_{i}))^{2}.
\end{equation*}
Here the minimum is taken over the set of all convex functions \(\psi : \Real^{d} \rightarrow \Real\).
Asymptotic and finite sample roperties of the convex LSE have been described in detail \cite{seijo2011}.
The equivalent, finite dimensional problem
\begin{equation}
    \label{eq:problem_statement}
    \underset{\btheta; \bxi_{1}, \ldots, \bxi_{n}}{\mathrm{minimize}}
    ~\frac{1}{2} \|\by - \btheta\|_{2}^{2}
    \quad
    \text{subject to}
    ~\theta_{j} + \langle{\bx_{i} - \bx_{j},~\bxi_{j}}\rangle \le \theta_{i}
\end{equation}
estimates function values \(\theta_{i} = \varphi_{n}(\bx_{i})\) and subgradients \(\bxi_{i} \in \Real^{d}\) at each data point.
The discretization \(\hat{\btheta}\) can be extended to a continuous, convex function on \(\Real^{d}\) by the affine interpolation scheme
\begin{equation*}
    \hat{\varphi}(\bx)
    =
    \max_{j=1,\ldots,n}\{\hat{\theta}_{j}
    + \langle{\bx - \bx_{j},~\hat{\bxi}_{j}}\rangle\}.
\end{equation*}
Further, one can restore smoothness by imposing Lipschitz constraints on each subgradient or applying postprocessing techniques \cite{mazumder2018}.

Yet, efficient computation remains a challenge.
Generic interior point methods do not scale well due to the \(n^{2}\) constraints.
Alternating direction method of multipliers (ADMM) has been proposed as an alternative that adequately exploits problem-specific structure \cite{mazumder2018}.
Empirical evidence suggests ADMM scales to problems on the order of \(n \approx 5000\) samples and is largely insensitive to the dimension \(d\) of a function's domain.
However, the convergence properties of 3-block ADMM algorithms are not characterized and so the quality of solutions cannot be readily assessed.
Moreover, ADMM is known to converge slowly to high precision.

These notes explore proximal distance algorithms tailored to convex regression.
We anticipate that the simplicity of updates, acceleration techniques, and well-established convergence theory will prove superior to ADMM in this context.

\section*{\center Setup and Structure of Constraints}

The convex regression problem is a quadratic program
\begin{equation*}
    \underset{\bz}{\mathrm{minimize}}
    ~f(\bA \bz) \quad
    \text{subject to}
    ~\bB \bz \le \boldsymbol{0},
\end{equation*}
where \(\bz = (\btheta, \bxi_{1}, \ldots, \bxi_{n})^{t}\) and \(\bA\) and \(\bB\) are compatible matrices to be determined.
Its solution is approximated by an unconstrained program in which the sublevel set constraint is replaced with the distance penalty \(q(\bz) = \frac{1}{2} \dist(\bB \bz, \Real_{-}^{p})^{2}\).
Explicitly, we pass to the optimization problem
\begin{equation}
    \label{eq:unconstrained_problem}
    \underset{\btheta, \bxi}{\mathrm{minimize}}
    ~h_{\rho}(\btheta, \bxi)
    =
    \underset{\btheta, \bxi}{\mathrm{minimize}}
    ~\frac{1}{2} \|\btheta - \by\|^{2}_{2}
    + \frac{\rho}{2} \dist(\bD \btheta + \bH \bxi, \Real_{-}^{p})^{2},
\end{equation}
based on the choices \(\bA = \begin{bmatrix}
    \bI_{n \times n} & \boldsymbol{0}_{n \times nd}
\end{bmatrix}\)
and \(\bB = \begin{bmatrix}
    \bD & \bH
\end{bmatrix}\).
Both the loss and penalty functions have easy-to-compute proximal maps.
Projection onto the non-positive orthant \(\Real_{-}^{p}\) is straightforward albeit \(p\) is order \(\mathcal{O}(n^{2})\).
Moreover, the distance penalty \(\dist(\bx, S)\) has the obvious majorization \(\|\bx - P_{S}(\bx)\|^{2}_{2}\).
This leads to a family of surrogates
\begin{equation}
    \label{eq:distance_majorization}
    g_{\rho}(\btheta, \bxi \mid \btheta_{n}, \bxi_{n})
    =
    \frac{1}{2} \|\btheta - \by\|^{2}_{2}
    +
    \frac{\rho}{2} \|\bD \btheta + \bH \bxi - P_{S}(\bD \btheta_{n} + \bH \bxi_{n})\|^{2}_{2}.
\end{equation}
The relevant gradient is of the form
\begin{equation*}
    \bq_{n}
    =
    \begin{bmatrix}
        \btheta - \by \\
        \boldsymbol{0}
    \end{bmatrix}
    +
    \rho \begin{bmatrix}
        \bD^{\ast} \\
        \bH^{\ast}
    \end{bmatrix}[\bD \btheta + \bH \bxi
        - P_{S}(\bD \btheta_{n} + \bH \bxi_{n})].
\end{equation*}

The remainder of this section is devoted to deriving explicit formulae for the actions of \(\bD\) and \(\bH\), their respective adjoints, and the required proximal maps that serve as ingredients for the various algorithms contained in this text.
Note that we have avoided describing entries of \(\bD\) and \(\bH\).
This is because there are two possible approaches: one with redundant constraints and one without.

\subsection*{Linear operators with redundant constraints}

This version has \(\bD\) as a \(n^{2} \times n\) matrix and \(\bH\) as a \(n^{2} \times nd\) matrix.
We take \(\bw = \vec(\bW)\) with \(\bW \in \Real^{n \times n}\).

Multiplication by \(\bD\) follows directly from the definition of the constraints:
\begin{equation}
    \label{eq:operator_D_redundant}
    [\bD \btheta]_{(i-1)n+j} = \theta_{j} - \theta_{i}.
\end{equation}
The Riesz representation theorem helps us understand multiplication by \(\bD^{\ast}\):
\begin{equation}
    \label{eq:operator_Dt_redundant}
    \begin{split}
        \langle{\bD \btheta, \bw}\rangle
        &=
        \sum_{i,j} (\bD \btheta)_{(i-1)n+j}~w_{(i-1)n+j} \\
        &=
        \sum_{i,j} (\theta_{j} - \theta_{i}) w_{ij} \\
        &=
        \langle{\bW^{\ast} \boldsymbol{1}, \btheta}\rangle
        -
        \langle{\bW \boldsymbol{1}, \btheta}\rangle \\
        &=
        \langle{(\bW^{\ast} - \bW) \boldsymbol{1}, \btheta}\rangle \\
        \implies
        \bD^{\ast} \bw
        &=
        (\bW^{\ast} - \bW) \boldsymbol{1}
    \end{split}
\end{equation}

Multiplication by \(\bH\) follows directly from the constraints.
The matrix is organized into columns \(\bH_{\cdot, j}\) of width \(d\), each operating on the blocks \(\bxi_{j}\) contained in \(\bxi\).
The \(j\)th block is of the form
\begin{equation}
    \label{eq:operator_H_redundant}
    \bH_{\cdot,j}
    =
    \begin{bmatrix}
        (\bx_{1} - \bx_{j})^{\ast} \\
        \vdots \\
        (\bx_{i} - \bx_{j})^{\ast} \\
        \vdots \\
        (\bx_{n} - \bx_{j})^{\ast} \\
    \end{bmatrix}
    \qquad
    j = 1,2,\ldots,n,
\end{equation}
so that \(\bH \bxi = \sum_{j}^{n} \bH_{\cdot,j} \bxi\).
This representation leads to a simple formula for \(\bH^{\ast}\):
\begin{equation}
    \label{eq:operator_Ht_redundant}
    \bH_{\cdot,j}^{\ast} \bw
    =
    \sum_{i=1}^{n} w_{ij} (\bx_{i} - \bx_{j}).
\end{equation}
Further note that \(\bH\) is skew-symmetric; that is, \(\bH = \bL - \bL^{\ast}\) with \(\bL\) a strictly lower-triangular matrix.

The Gram matrix \(\bB^{\ast} \bB \in \Real^{n(1+d) \times n(1+d)}\) consists of the blocks
\begin{equation*}
    \begin{bmatrix}
        \bD^{\ast} \bD
        & \bD^{\ast} \bH_{\cdot,1}
        & \bD^{\ast} \bH_{\cdot,2}
        & \cdots
        & \bD^{\ast} \bH_{\cdot,n} \\
        \bH_{\cdot,1}^{\ast} \bD
        & \bH_{\cdot,1}^{\ast} \bH_{\cdot,1}
        & \boldsymbol{0}_{d \times d}
        & \cdots
        & \boldsymbol{0}_{d \times d} \\
        \bH_{\cdot,2}^{\ast} \bD
        & \boldsymbol{0}_{d \times d}
        & \bH_{\cdot,2}^{\ast} \bH_{\cdot,2}
        & \ddots
        & \vdots \\
        \vdots
        & \vdots
        & \ddots
        & \ddots
        & \boldsymbol{0}_{d \times d} \\
       \bH_{\cdot,n}^{\ast} \bD
        & \boldsymbol{0}_{d \times d}
        & \cdots
        & \boldsymbol{0}_{d \times d}
        &\bH_{\cdot,n}^{\ast} \bH_{\cdot,n}
    \end{bmatrix},
\end{equation*}
which implies easy formulas for multiplying \((\btheta, \bxi)\) in terms of \(n+1\) blocks.
These results may provide a stencils for extending proximal distance algorithms to a distributed computing environment.

\subsection*{Linear operators without redundancy}

The constraints \(\theta_{j} - \theta_{i} + \langle{\bx_{i} - \bx_{j}, \bxi_{j}}\rangle \le 0\) with \(i = j\) do not encode any information.
We can therefore reduce dimensions in our linear operators.
Specifically, we have \(\bD\) as a \(n(n-1) \times n\) matrix and \(\bH\) as a \(n(n-1) \times nd\) matrix.
This should be the optimal representation in terms of memory requirements, as we shall soon observe.

Once again, multiplication by \(bD\) is straightforward based on the constraint definitions:
\begin{equation}
    \label{eq:operator_D_efficient}
    [\bD \btheta]_{k}
    =
    \theta_{j} - \theta_{i}
\qquad i \neq j, \\
\end{equation}
with \(k \in \{1, 2, \ldots, \binom{n}{2}\}\) counting the unique comparisons.
The blocks of \(\bH\) are now given by
\begin{equation}
    \bH_{(j-1)\binom{n}{2}+k,j}
    =
    \begin{bmatrix}
        \boldsymbol{0} \\
        \vdots \\
        (\bx_{i} - \bx_{j})^{\ast} \\
        \vdots \\
        \boldsymbol{0}
    \end{bmatrix}
    \qquad i \neq j.
\end{equation}
For example, only the first \(\binom{n}{2}\) entries of \(\bH_{\cdot,1}\) are non-zero.
This pattern makes \(\bH\) block diagonal.

The non-trivial operators elude me at the moment\ldots

\subsection*{Relevant proximal maps}

\begin{equation*}
    \begin{split}
        \prox_{f}(\bA[\bz-\tilde{\by}])
        =
        \prox_{f}(\btheta - \by)
        &=
        \begin{cases}
            \by + \left(1 - \frac{1}{2 \|\btheta - \by\|}\right) (\btheta - \by),
                & \|\btheta - \by\| \ge \frac{1}{2} \\
            \by,
                & \text{otherwise}
        \end{cases} \\
        \prox_{\rho^{-1} g}(\bb)
        &=
        \frac{1}{1 + \rho} \bb
            + \frac{\rho}{1 + \rho} P_{S}(\bb).
    \end{split}
\end{equation*}

\section*{\center Algorithm 1: Proximal Distance}
Exploiting the stationarity condition \(\bq_{n} = \boldsymbol{0}\) produces the MM algorithm
\begin{equation*}
    \begin{bmatrix}
        \btheta_{n+1} \\
        \bxi_{n+1}
    \end{bmatrix}
    =
    \rho (\bI + \rho \bB^{\ast} \bB)^{-1}
    \begin{bmatrix}
        \bD^{\ast} \\
        \bH^{\ast}
    \end{bmatrix}
    P_{S}(\bD \btheta_{n} + \bH \bxi_{n}).
\end{equation*}
Importantly, the matrix \(\bB^{\ast} \bB\) is typically positive semidefinite.
This update is handled well using the method of conjugate gradients or LSMR because \(\bI + \bB^{\ast} \bB\) is sparse.

\section*{\center Algorithm 2: Proximal Distance + Steepest Descent}

Alternatively, we can exploit tangency to construct a steepest descent algorithm
\begin{equation*}
    \bz_{n+1}
    =
    \bz_{n} - \gamma_{n} \nabla g_{\rho}(\bz_{n} \mid \bz_{n})
    =
    \bz_{n} - \gamma_{n} \nabla h_{\rho}(\bz_{n})
\end{equation*}
Explicitly, the update to each block follows from the formulas
\begin{align*}
    \btheta_{n+1}
    &=
    \btheta_{n} - \gamma_{n}\left[
        \btheta_{n} - \by
        + \rho \bD^{\ast} \left(\bD \btheta_{n}
        + \bD^{\ast} \bH \bxi_{n} - \bw
        \right)
    \right] \\
    \bxi_{n+1,j}
    &=
    \bxi_{n,j} - \gamma_{n} \rho \bH_{\cdot,j}^{\ast}\left[
        \bD \btheta_{n}
        + \bH_{\cdot,j} \bxi_{n,j}
        - \bW_{\cdot,j}
    \right],
\end{align*}
where \(\vec(\bW) = (\bB \bz_{n})_{-}\) accounts for the requisite projection.
Because the gradient is linear in \(\bz\) we can solve for the optimal step size by maximizing \(g_{\rho}(\bz_{n} - \gamma \bq \mid \bz_{n})\) with respect to \(\gamma\).
Straightforward application of the chain rule yields
\begin{equation*}
    \gamma_{n}
    =
    \frac{\|\bv_{n}\|^{2}}{\|\bA \bv_{n}\|^{2} + \rho \|\bB \bv_{n}\|^{2}}.
\end{equation*}

\section*{\center Algorithm 3: Operator Splitting}

In view of the splitting property of proximal maps, it is tempting to pass to the quadratic program
\begin{equation*}
    \mathrm{minmize}~h_{\rho}(\bz, \bb) =
        f(\bz) + \rho~g(\bb)
    \quad \text{subject to}~\bb = \bB \bz.
\end{equation*}
(Check!) If \(\bB^{\ast} \bB\) is positive semi-definite, then this problem can be solved by iterating the proximal map for \(h_{\rho}\) while maintaining \((\bz_{n}, \bb_{n})\) within the subspace \(V = \{(\bz, \bb) : \bb = \bB \bz\}\).
This may be achieved by the projection
\begin{equation*}
    P_{V}(\bz, \bb)
    =
    \begin{bmatrix}
        \bz \\
        \bb
    \end{bmatrix}
    +
    \begin{bmatrix}
        \bB^{\ast} \\
        -\bI
    \end{bmatrix}
    (\bI + \bB \bB^{\ast})^{-1} (\bb - \bB \bz).
\end{equation*}
Our iteration scheme is as follows:
\begin{enumerate}
    \item \((\bz_{n}^{+}, \bb_{n}^{+}) = P_{V}(\bz_{n}, \bb_{n})\)
    \item \(\bz_{n+1} = \prox_{f}(\bz_{n}^{+})\)
    \item \(\bb_{n+1} = \prox_{\rho^{-1}g}(\bb)\)
\end{enumerate}
% total variation de-noising
% convex clustering

%%%%% BIBLIOGRAPHY %%%%%
\begin{thebibliography}{99}
    \bibitem{seijo2011}
    Seijo E, Sen B (2011) {Nonparametric least squares estimation of a multivariate convex regression function}. {\it The Annals of Statistics}. 1633-1657
    
    \bibitem{langeMM2016}
    Lange, K (2016) {MM Optimization Algorithms}. {\it SIAM}.

    \bibitem{mazumder2018}
    Mazumder R, Choudhury A, Iyengar G, Sen B (2018) {A computational framework for multivariate convex regression and its variants}. {\it Journal of the American Statistical Association}. 318-331
\end{thebibliography}
\end{document}