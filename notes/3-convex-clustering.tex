\documentclass[11pt]{article}

\usepackage{amsbsy,amsthm,amsmath,amssymb}
%\usepackage{graphicx,natbib,booktabs}
\usepackage{graphicx}
\usepackage[margin=1in]{geometry}
\usepackage{subfig}
\usepackage{algorithm}
\usepackage{algpseudocode}
\usepackage{paralist}

\newtheorem{proposition}{Proposition}
\newtheorem{lemma}{Lemma}
\newtheorem{example}{Example}
\newtheorem{definition}{Definition}
\def\E{\mathop{\rm E\,\!}\nolimits}
\def\Var{\mathop{\rm Var}\nolimits}
\def\Cov{\mathop{\rm Cov}\nolimits}
\def\den{\mathop{\rm den}\nolimits}
\def\midd{\mathop{\,|\,}\nolimits}
\def\sgn{\mathop{\rm sgn}\nolimits}
\def\vec{\mathop{\rm vec}\nolimits}
\def\sinc{\mathop{\rm sinc}\nolimits}
\def\curl{\mathop{\rm curl}\nolimits}
\def\div{\mathop{\rm div}\nolimits}
\def\tr{\mathop{\rm tr}\nolimits}
\def\len{\mathop{\rm len}\nolimits}
\def\dist{\mathop{\rm dist}\nolimits}
\def\prox{\mathop{\rm prox}\nolimits}
\def\amp{\mathop{\:\:\,}\nolimits}
\def\Real{\mathop{\mathbb{R}}\nolimits}
\def\dom{\mathop{\bf dom}\nolimits}
\def\argmin{\mathop{\rm argmin}\nolimits}
\def\argmax{\mathop{\rm argmax}\nolimits}
\newcommand{\svskip}{\vspace{1.75mm}}
\newcommand{\ba}{\boldsymbol{a}}
\newcommand{\bb}{\boldsymbol{b}}
\newcommand{\bc}{\boldsymbol{c}}
\newcommand{\bd}{\boldsymbol{d}}
\newcommand{\be}{\boldsymbol{e}}
\newcommand{\bff}{\boldsymbol{f}}
\newcommand{\bg}{\boldsymbol{g}}
\newcommand{\bh}{\boldsymbol{h}}
\newcommand{\bi}{\boldsymbol{i}}
\newcommand{\bj}{\boldsymbol{j}}
\newcommand{\bk}{\boldsymbol{k}}
\newcommand{\bl}{\boldsymbol{l}}
\newcommand{\bm}{\boldsymbol{m}}
\newcommand{\bn}{\boldsymbol{n}}
\newcommand{\bo}{\boldsymbol{o}}
\newcommand{\bp}{\boldsymbol{p}}
\newcommand{\bq}{\boldsymbol{q}}
\newcommand{\br}{\boldsymbol{r}}
\newcommand{\bs}{\boldsymbol{s}}
\newcommand{\bt}{\boldsymbol{t}}
\newcommand{\bu}{\boldsymbol{u}}
\newcommand{\bv}{\boldsymbol{v}}
\newcommand{\bw}{\boldsymbol{w}}
\newcommand{\bx}{\boldsymbol{x}}
\newcommand{\by}{\boldsymbol{y}}
\newcommand{\bz}{\boldsymbol{z}}
\newcommand{\bA}{\boldsymbol{A}}
\newcommand{\bB}{\boldsymbol{B}}
\newcommand{\bC}{\boldsymbol{C}}
\newcommand{\bD}{\boldsymbol{D}}
\newcommand{\bE}{\boldsymbol{E}}
\newcommand{\bF}{\boldsymbol{F}}
\newcommand{\bG}{\boldsymbol{G}}
\newcommand{\bH}{\boldsymbol{H}}
\newcommand{\bI}{\boldsymbol{I}}
\newcommand{\bJ}{\boldsymbol{J}}
\newcommand{\bK}{\boldsymbol{K}}
\newcommand{\bL}{\boldsymbol{L}}
\newcommand{\bM}{\boldsymbol{M}}
\newcommand{\bN}{\boldsymbol{N}}
\newcommand{\bO}{\boldsymbol{O}}
\newcommand{\bP}{\boldsymbol{P}}
\newcommand{\bQ}{\boldsymbol{Q}}
\newcommand{\bR}{\boldsymbol{R}}
\newcommand{\bS}{\boldsymbol{S}}
\newcommand{\bT}{\boldsymbol{T}}
\newcommand{\bU}{\boldsymbol{U}}
\newcommand{\bV}{\boldsymbol{V}}
\newcommand{\bW}{\boldsymbol{W}}
\newcommand{\bX}{\boldsymbol{X}}
\newcommand{\bY}{\boldsymbol{Y}}
\newcommand{\bZ}{\boldsymbol{Z}}
\newcommand{\balpha}{\boldsymbol{\alpha}}
\newcommand{\bbeta}{\boldsymbol{\beta}}
\newcommand{\bgamma}{\boldsymbol{\gamma}}
\newcommand{\bdelta}{\boldsymbol{\delta}}
\newcommand{\bepsilon}{\boldsymbol{\epsilon}}
\newcommand{\blambda}{\boldsymbol{\lambda}}
\newcommand{\bmu}{\boldsymbol{\mu}}
\newcommand{\bnu}{\boldsymbol{\nu}}
\newcommand{\bphi}{\boldsymbol{\phi}}
\newcommand{\bpi}{\boldsymbol{\pi}}
\newcommand{\bsigma}{\boldsymbol{\sigma}}
\newcommand{\btheta}{\boldsymbol{\theta}}
\newcommand{\bomega}{\boldsymbol{\omega}}
\newcommand{\bxi}{\boldsymbol{\xi}}
\newcommand{\bGamma}{\boldsymbol{\rho}}
\newcommand{\bDelta}{\boldsymbol{\Delta}}
\newcommand{\bTheta}{\boldsymbol{\Theta}}
\newcommand{\bLambda}{\boldsymbol{\Lambda}}
\newcommand{\bXi}{\boldsymbol{\Xi}}
\newcommand{\bPi}{\boldsymbol{\Pi}}
\newcommand{\bOmega}{\boldsymbol{\Omega}}
\newcommand{\bUpsilon}{\boldsymbol{\Upsilon}}
\newcommand{\bPhi}{\boldsymbol{\Phi}}
\newcommand{\bPsi}{\boldsymbol{\Psi}}
\newcommand{\bSigma}{\boldsymbol{\Sigma}}
%% Matrix-matrix operations
\newcommand{\Kron}{\otimes} %Kronecker
%\newcommand{\Khat}{\odot} %Khatri-Rao
\newcommand{\Hada}{\ast} %Hadamard
%\newcommand{\Divide}{\varoslash}
%% Norms
\newcommand{\abs}[1]{\lvert{#1}\rvert}


\title{Example 3: Convex Clustering}
\author{}
\date{}

\begin{document}
\maketitle

Convex clustering of \(n\) samples based on \(d\) features can be formulated in terms of the regularized objective
\begin{equation}
    \label{eq:regularized-objective}
    F_{\gamma}(\bX)
    =
    \frac{1}{2} \|\bX - \bU\|_{F}^{2}
    +
    \gamma \sum_{i > j} w_{ij} \|\bX (\be_{i} - \be_{j})\|,
\end{equation}
where \(\bU \in \Real^{d \times n}\) encodes the input data, columns of \(\bX \in \Real^{d \times n}\) represent cluster assignments, and \(\be_{k} \in \Real^{n}\) is a standard basis vector.
Interpreting $(w_{ij})$ as an adjacency matrix, we assume the graph is connected otherwise the objective splits over connected components.
Letting \(S_{\nu}\) denote the set of \(\nu\) block sparse vectors, we pass to the proximal distance framework by considering the penalized objective
\begin{align*}
    h_{\rho}(\bX)
    &=
    \frac{1}{2}\|\bX - \bU\|_{F}^{2}
    +
    \frac{\rho}{2} \dist(\bD \bX, S_{\nu})^{2},
\end{align*}
where matrix $\bD$ acts as a forward difference operator on columns.
Taking $\bx = \vec(\bX)$ and mildly abusing notation, its surrogate is
\begin{equation*}
    g_{\rho}(\bx \mid \bx_{n})
    =
    \frac{1}{2}\|\bx - \bu\|_{2}^{2}
    +
    \frac{\rho}{2} \|\bD \bx - \mathcal{P}_{\nu}(\bD \bx_{n})\|^{2}.
\end{equation*}

\section*{\center Blockwise Sparse Projection}

The projection $\mathcal{P}_{\nu}$ maps a matrix to a sparse representation with $\nu$ non-zero columns (or blocks in the case of the vectorized version).
In the context of clustering, sparsity permits a maximum of $\nu$ violations in consensus constraints $\bx_{i} = \bx_{j}$.
Letting $\Delta_{ij} = \|\bx_{i} - \bx_{j}\|$ denote pairwise distances, we define the projection along blocks $\bv_{k}$ as
\begin{align*}
    \mathcal{P}_{\nu}(\bv_{k})
    =
    \begin{cases}
        \bv_{k}, & \text{if}~\Delta_{k} \in \{\Delta_{(m)}, \Delta_{(m-1)}, \ldots \Delta_{(m-\nu+1)}\} \\
        \boldsymbol{0}, & \text{otherwise}.
    \end{cases}
\end{align*}
Concretely, the magnitude of a difference $\bv_{k}$ must be within the top $\nu$ distances.
An alternative, helpful definition is based on the smallest distances
\begin{align*}
    \mathcal{P}_{\nu}(\bv_{k})
    =
    \begin{cases}
        \boldsymbol{0}, & \text{if}~\Delta_{k} \in \{\Delta_{(1)}, \Delta_{(m-1)}, \ldots \Delta_{(\nu)}\} \\
        \bv_{k}, & \text{otherwise}
    \end{cases}
\end{align*}
Thus, it is enough to find a pivot $\Delta_{(m-\nu+1)}$ or $\Delta_{(\nu)}$.
Because the sparsity parameter $\nu$ has a finite range in $\{0,1,2,\ldots,\binom{n}{2}\}$ one can exploit symmetry to reduce the best/average computational complexity in a search procedure.
Internally, this projection is implemented using a partial sorting algorithm based on quicksort.
Note that this projetion operator is set-valued in general.

\section*{\center Algorithm Maps}

\subsection*{MM}
Rewrite the surrogate explicitly a least squares problem minimizing $\|\bA \bx - \bb_{n}\|^{2}_{2}$:
\begin{equation*}
  \bx_{n+1} = \underset{\bx}{\argmin} \frac{1}{2} \left\|
    \begin{bmatrix}
      \bI \\
      \sqrt{\rho} \bD
    \end{bmatrix} \bx
    -
    \begin{bmatrix}
      \bu \\
      \sqrt{\rho} \mathcal{P}_{\nu}(\bD \bx_{n})
    \end{bmatrix}
  \right\|_{2}^{2}
\end{equation*}

\subsection*{Steepest Descent}

The updates $\bx_{n+1} = \bx_{n} - \gamma_{n} \nabla h_{\rho}(\bx_{n})$ admit an exact solution for the line search parameter $\gamma_{n}$.
Taking $\bq_{n} = \nabla h_{\rho}(\bx_{n})$ as the gradient we have
\begin{align*}
  \bq_{n}
  &= (\bx_{n} - \bu) + \rho \bD^{t} [\bD \bx_{n} - \mathcal{P}_{\nu}(\bD \bx_{n})] \\
  \gamma_{n}
  &=
  \frac{\|\bq_{n}\|^{2}}{\|\bq_{n}\|^{2} + \rho \|\bD \bq_{n}\|^{2}}.
\end{align*}
Note that blocks in $[\bD \bx_{n} - \mathcal{P}_{\nu}(\bD\bx_{n})]_{k}$ are equal to $\boldsymbol{0}$ whenever the projection of $[\bD \bx_{n}]_{k}$ is non-zero.

\subsection*{ADMM}

Take $\by$ as the dual variable and $\blambda$ as scaled multipliers.
Minimizing the $\bx$ block involves solving a single linear system:
\begin{align*}
    \bx_{n+1}
    &=
    \underset{\bx}{\argmin} \frac{1}{2} \left\|
        \begin{bmatrix}
        \bI \\
        \sqrt{\mu} \bD
        \end{bmatrix} \bx
        -
        \begin{bmatrix}
        \bu \\
        \sqrt{\mu} (\blambda_{n} - \by_{n})
        \end{bmatrix}
    \right\|_{2}^{2} \\
    \by_{n+1}
    &= \frac{\alpha}{1+\alpha} \mathcal{P}_{\nu}(\bz_{n}) + \frac{1}{1+\alpha} \bz_{n};
    \qquad \bz_{n} = \bD \bx_{n+1} + \blambda_{n},~\alpha = \rho / \mu
    \end{align*}
Multipliers follow the standard update.

\begin{thebibliography}{1}
    \bibitem{chi2015}
    Chi, E. C., Lange, K. (2015). {Splitting Methods for Convex Clustering}. {Journal of Computational and Graphical Statistics}, 24(4), 994–1013.
\end{thebibliography}
\end{document}